
%% bare_conf.tex
%% V1.3
%% 2007/01/11
%% by Michael Shell
%% See:
%% http://www.michaelshell.org/
%% for current contact information.
%%
%% This is a skeleton file demonstrating the use of IEEEtran.cls
%% (requires IEEEtran.cls version 1.7 or later) with an IEEE conference paper.
%%
%% Support sites:
%% http://www.michaelshell.org/tex/ieeetran/
%% http://www.ctan.org/tex-archive/macros/latex/contrib/IEEEtran/
%% and
%% http://www.ieee.org/

%%*************************************************************************
%% Legal Notice:
%% This code is offered as-is without any warranty either expressed or
%% implied; without even the implied warranty of MERCHANTABILITY or
%% FITNESS FOR A PARTICULAR PURPOSE!
%% User assumes all risk.
%% In no event shall IEEE or any contributor to this code be liable for
%% any damages or losses, including, but not limited to, incidental,
%% consequential, or any other damages, resulting from the use or misuse
%% of any information contained here.
%%
%% All comments are the opinions of their respective authors and are not
%% necessarily endorsed by the IEEE.
%%
%% This work is distributed under the LaTeX Project Public License (LPPL)
%% ( http://www.latex-project.org/ ) version 1.3, and may be freely used,
%% distributed and modified. A copy of the LPPL, version 1.3, is included
%% in the base LaTeX documentation of all distributions of LaTeX released
%% 2003/12/01 or later.
%% Retain all contribution notices and credits.
%% ** Modified files should be clearly indicated as such, including  **
%% ** renaming them and changing author support contact information. **
%%
%% File list of work: IEEEtran.cls, IEEEtran_HOWTO.pdf, bare_adv.tex,
%%                    bare_conf.tex, bare_jrnl.tex, bare_jrnl_compsoc.tex
%%*************************************************************************

% *** Authors should verify (and, if needed, correct) their LaTeX system  ***
% *** with the testflow diagnostic prior to trusting their LaTeX platform ***
% *** with production work. IEEE's font choices can trigger bugs that do  ***
% *** not appear when using other class files.                            ***
% The testflow support page is at:
% http://www.michaelshell.org/tex/testflow/



% Note that the a4paper option is mainly intended so that authors in
% countries using A4 can easily print to A4 and see how their papers will
% look in print - the typesetting of the document will not typically be
% affected with changes in paper size (but the bottom and side margins will).
% Use the testflow package mentioned above to verify correct handling of
% both paper sizes by the user's LaTeX system.
%
% Also note that the "draftcls" or "draftclsnofoot", not "draft", option
% should be used if it is desired that the figures are to be displayed in
% draft mode.
%
\documentclass[conference]{IEEEtran}
% Add the compsoc option for Computer Society conferences.
%
% If IEEEtran.cls has not been installed into the LaTeX system files,
% manually specify the path to it like:
% \documentclass[conference]{../sty/IEEEtran}





% Some very useful LaTeX packages include:
% (uncomment the ones you want to load)


% *** MISC UTILITY PACKAGES ***
%
%\usepackage{ifpdf}
% Heiko Oberdiek's ifpdf.sty is very useful if you need conditional
% compilation based on whether the output is pdf or dvi.
% usage:
% \ifpdf
%   % pdf code
% \else
%   % dvi code
% \fi
% The latest version of ifpdf.sty can be obtained from:
% http://www.ctan.org/tex-archive/macros/latex/contrib/oberdiek/
% Also, note that IEEEtran.cls V1.7 and later provides a builtin
% \ifCLASSINFOpdf conditional that works the same way.
% When switching from latex to pdflatex and vice-versa, the compiler may
% have to be run twice to clear warning/error messages.






% *** CITATION PACKAGES ***
%
%\usepackage{cite}
% cite.sty was written by Donald Arseneau
% V1.6 and later of IEEEtran pre-defines the format of the cite.sty package
% \cite{} output to follow that of IEEE. Loading the cite package will
% result in citation numbers being automatically sorted and properly
% "compressed/ranged". e.g., [1], [9], [2], [7], [5], [6] without using
% cite.sty will become [1], [2], [5]--[7], [9] using cite.sty. cite.sty's
% \cite will automatically add leading space, if needed. Use cite.sty's
% noadjust option (cite.sty V3.8 and later) if you want to turn this off.
% cite.sty is already installed on most LaTeX systems. Be sure and use
% version 4.0 (2003-05-27) and later if using hyperref.sty. cite.sty does
% not currently provide for hyperlinked citations.
% The latest version can be obtained at:
% http://www.ctan.org/tex-archive/macros/latex/contrib/cite/
% The documentation is contained in the cite.sty file itself.






% *** GRAPHICS RELATED PACKAGES ***
%
\ifCLASSINFOpdf
   \usepackage[pdftex]{graphicx}
  % declare the path(s) where your graphic files are
   \graphicspath{{./png/}{./jpg/}}
  % and their extensions so you won't have to specify these with
  % every instance of \includegraphics
   \DeclareGraphicsExtensions{.pdf,.jpeg,.png,.jpg}
\else
  % or other class option (dvipsone, dvipdf, if not using dvips). graphicx
  % will default to the driver specified in the system graphics.cfg if no
  % driver is specified.
  % \usepackage[dvips]{graphicx}
  % declare the path(s) where your graphic files are
  % \graphicspath{{../eps/}}
  % and their extensions so you won't have to specify these with
  % every instance of \includegraphics
  % \DeclareGraphicsExtensions{.eps}
\fi
% graphicx was written by David Carlisle and Sebastian Rahtz. It is
% required if you want graphics, photos, etc. graphicx.sty is already
% installed on most LaTeX systems. The latest version and documentation can
% be obtained at:
% http://www.ctan.org/tex-archive/macros/latex/required/graphics/
% Another good source of documentation is "Using Imported Graphics in
% LaTeX2e" by Keith Reckdahl which can be found as epslatex.ps or
% epslatex.pdf at: http://www.ctan.org/tex-archive/info/
%
% latex, and pdflatex in dvi mode, support graphics in encapsulated
% postscript (.eps) format. pdflatex in pdf mode supports graphics
% in .pdf, .jpeg, .png and .mps (metapost) formats. Users should ensure
% that all non-photo figures use a vector format (.eps, .pdf, .mps) and
% not a bitmapped formats (.jpeg, .png). IEEE frowns on bitmapped formats
% which can result in "jaggedy"/blurry rendering of lines and letters as
% well as large increases in file sizes.
%
% You can find documentation about the pdfTeX application at:
% http://www.tug.org/applications/pdftex





% *** MATH PACKAGES ***
%
%\usepackage[cmex10]{amsmath}
% A popular package from the American Mathematical Society that provides
% many useful and powerful commands for dealing with mathematics. If using
% it, be sure to load this package with the cmex10 option to ensure that
% only type 1 fonts will utilized at all point sizes. Without this option,
% it is possible that some math symbols, particularly those within
% footnotes, will be rendered in bitmap form which will result in a
% document that can not be IEEE Xplore compliant!
%
% Also, note that the amsmath package sets \interdisplaylinepenalty to 10000
% thus preventing page breaks from occurring within multiline equations. Use:
%\interdisplaylinepenalty=2500
% after loading amsmath to restore such page breaks as IEEEtran.cls normally
% does. amsmath.sty is already installed on most LaTeX systems. The latest
% version and documentation can be obtained at:
% http://www.ctan.org/tex-archive/macros/latex/required/amslatex/math/

\usepackage{siunitx}





% *** SPECIALIZED LIST PACKAGES ***
%
%\usepackage{algorithmic}
% algorithmic.sty was written by Peter Williams and Rogerio Brito.
% This package provides an algorithmic environment fo describing algorithms.
% You can use the algorithmic environment in-text or within a figure
% environment to provide for a floating algorithm. Do NOT use the algorithm
% floating environment provided by algorithm.sty (by the same authors) or
% algorithm2e.sty (by Christophe Fiorio) as IEEE does not use dedicated
% algorithm float types and packages that provide these will not provide
% correct IEEE style captions. The latest version and documentation of
% algorithmic.sty can be obtained at:
% http://www.ctan.org/tex-archive/macros/latex/contrib/algorithms/
% There is also a support site at:
% http://algorithms.berlios.de/index.html
% Also of interest may be the (relatively newer and more customizable)
% algorithmicx.sty package by Szasz Janos:
% http://www.ctan.org/tex-archive/macros/latex/contrib/algorithmicx/




% *** ALIGNMENT PACKAGES ***
%
%\usepackage{array}
% Frank Mittelbach's and David Carlisle's array.sty patches and improves
% the standard LaTeX2e array and tabular environments to provide better
% appearance and additional user controls. As the default LaTeX2e table
% generation code is lacking to the point of almost being broken with
% respect to the quality of the end results, all users are strongly
% advised to use an enhanced (at the very least that provided by array.sty)
% set of table tools. array.sty is already installed on most systems. The
% latest version and documentation can be obtained at:
% http://www.ctan.org/tex-archive/macros/latex/required/tools/


%\usepackage{mdwmath}
%\usepackage{mdwtab}
% Also highly recommended is Mark Wooding's extremely powerful MDW tools,
% especially mdwmath.sty and mdwtab.sty which are used to format equations
% and tables, respectively. The MDWtools set is already installed on most
% LaTeX systems. The lastest version and documentation is available at:
% http://www.ctan.org/tex-archive/macros/latex/contrib/mdwtools/


% IEEEtran contains the IEEEeqnarray family of commands that can be used to
% generate multiline equations as well as matrices, tables, etc., of high
% quality.


%\usepackage{eqparbox}
% Also of notable interest is Scott Pakin's eqparbox package for creating
% (automatically sized) equal width boxes - aka "natural width parboxes".
% Available at:
% http://www.ctan.org/tex-archive/macros/latex/contrib/eqparbox/





% *** SUBFIGURE PACKAGES ***
%\usepackage[tight,footnotesize]{subfigure}
% subfigure.sty was written by Steven Douglas Cochran. This package makes it
% easy to put subfigures in your figures. e.g., "Figure 1a and 1b". For IEEE
% work, it is a good idea to load it with the tight package option to reduce
% the amount of white space around the subfigures. subfigure.sty is already
% installed on most LaTeX systems. The latest version and documentation can
% be obtained at:
% http://www.ctan.org/tex-archive/obsolete/macros/latex/contrib/subfigure/
% subfigure.sty has been superceeded by subfig.sty.



%\usepackage[caption=false]{caption}
%\usepackage[font=footnotesize]{subfig}
% subfig.sty, also written by Steven Douglas Cochran, is the modern
% replacement for subfigure.sty. However, subfig.sty requires and
% automatically loads Axel Sommerfeldt's caption.sty which will override
% IEEEtran.cls handling of captions and this will result in nonIEEE style
% figure/table captions. To prevent this problem, be sure and preload
% caption.sty with its "caption=false" package option. This is will preserve
% IEEEtran.cls handing of captions. Version 1.3 (2005/06/28) and later
% (recommended due to many improvements over 1.2) of subfig.sty supports
% the caption=false option directly:
%\usepackage[caption=false,font=footnotesize]{subfig}
%
% The latest version and documentation can be obtained at:
% http://www.ctan.org/tex-archive/macros/latex/contrib/subfig/
% The latest version and documentation of caption.sty can be obtained at:
% http://www.ctan.org/tex-archive/macros/latex/contrib/caption/




% *** FLOAT PACKAGES ***
%
%\usepackage{fixltx2e}
% fixltx2e, the successor to the earlier fix2col.sty, was written by
% Frank Mittelbach and David Carlisle. This package corrects a few problems
% in the LaTeX2e kernel, the most notable of which is that in current
% LaTeX2e releases, the ordering of single and double column floats is not
% guaranteed to be preserved. Thus, an unpatched LaTeX2e can allow a
% single column figure to be placed prior to an earlier double column
% figure. The latest version and documentation can be found at:
% http://www.ctan.org/tex-archive/macros/latex/base/



%\usepackage{stfloats}
% stfloats.sty was written by Sigitas Tolusis. This package gives LaTeX2e
% the ability to do double column floats at the bottom of the page as well
% as the top. (e.g., "\begin{figure*}[!b]" is not normally possible in
% LaTeX2e). It also provides a command:
%\fnbelowfloat
% to enable the placement of footnotes below bottom floats (the standard
% LaTeX2e kernel puts them above bottom floats). This is an invasive package
% which rewrites many portions of the LaTeX2e float routines. It may not work
% with other packages that modify the LaTeX2e float routines. The latest
% version and documentation can be obtained at:
% http://www.ctan.org/tex-archive/macros/latex/contrib/sttools/
% Documentation is contained in the stfloats.sty comments as well as in the
% presfull.pdf file. Do not use the stfloats baselinefloat ability as IEEE
% does not allow \baselineskip to stretch. Authors submitting work to the
% IEEE should note that IEEE rarely uses double column equations and
% that authors should try to avoid such use. Do not be tempted to use the
% cuted.sty or midfloat.sty packages (also by Sigitas Tolusis) as IEEE does
% not format its papers in such ways.





% *** PDF, URL AND HYPERLINK PACKAGES ***
%
%\usepackage{url}
% url.sty was written by Donald Arseneau. It provides better support for
% handling and breaking URLs. url.sty is already installed on most LaTeX
% systems. The latest version can be obtained at:
% http://www.ctan.org/tex-archive/macros/latex/contrib/misc/
% Read the url.sty source comments for usage information. Basically,
% \url{my_url_here}.





% *** Do not adjust lengths that control margins, column widths, etc. ***
% *** Do not use packages that alter fonts (such as pslatex).         ***
% There should be no need to do such things with IEEEtran.cls V1.6 and later.
% (Unless specifically asked to do so by the journal or conference you plan
% to submit to, of course. )


% correct bad hyphenation here
\hyphenation{op-tical net-works semi-conduc-tor}


\begin{document}
%
% paper title
% can use linebreaks \\ within to get better formatting as desired
\title{Yubiduino: TOTP on an Arduino}


% author names and affiliations
% use a multiple column layout for up to three different
% affiliations
\author{\IEEEauthorblockN{Jessica Fleck, Brandon Mills, Paul Tela}
\IEEEauthorblockA{Department of Computer Science and Engineering \\
The Ohio State University \\
Columbus, OH 43210 \\
\{fleck.48, mills.511, tela.3\}@osu.edu}}

% make the title area
\maketitle


\begin{abstract}
Two-factor authorization (2FA) is a more secure method for authenticating users during login. Many 2FA systems, including the one used by the popular Gmail service, follow the Time-based One Time Password (TOTP) algorithm standardized in RFC 6238. Per the standard, the client and the server agree on a shared secret key and starting timestamp at setup. Then, each time the user attempts to log in, the algorithm generates an ephemeral numeric passcode by running a cryptographic function with the shared secret key and the number of elapsed time intervals since the starting timestamp as inputs. However, this system requires a user to obtain and manually enter another unique password every time they log in. Inspired by the commercial YubiKey multi-factor authorization key, we designed and built a device, called a Yubiduino, that calculates and automatically enters the one-time password on the user's behalf. The device consists of an Arduino Due microcontroller, a DS3231 real time clock with a battery backup, a microSD card for persistent storage, and a button that the user presses to activate the algorithm. After setup, the user has only to plug the Yubiduino into a USB port, select the field in the login form for the one-time password, and press the button, and the Yubiduino will calculate and type the password automatically.

\end{abstract}
% IEEEtran.cls defaults to using nonbold math in the Abstract.
% This preserves the distinction between vectors and scalars. However,
% if the conference you are submitting to favors bold math in the abstract,
% then you can use LaTeX's standard command \boldmath at the very start
% of the abstract to achieve this. Many IEEE journals/conferences frown on
% math in the abstract anyway.

% no keywords




% For peer review papers, you can put extra information on the cover
% page as needed:
% \ifCLASSOPTIONpeerreview
% \begin{center} \bfseries EDICS Category: 3-BBND \end{center}
% \fi
%
% For peerreview papers, this IEEEtran command inserts a page break and
% creates the second title. It will be ignored for other modes.
\IEEEpeerreviewmaketitle



\section{Introduction}
% no \IEEEPARstart
Internet services permeate every facet of modern life. Communication, finances, and even our very identities all live online. Through the power of data, Google, Facebook, and dating company OkCupid know more about their users than their users know about themselves. This trove of personal and financial information is an enticing treasure to malfeasants, who contrive ever more sophisticated attacks through backdoors and social engineering. Multi-factor authentication stands in the way of an attempted account hijacking by verifying identity through a combination of things only the user \textit{knows}, \textit{has}, or \textit{is}. In the case of the Yubiduino, the user's standard login password is something only the user \textit{knows}, and the one-time password is generated from the secret key that nobody but the user \textit{has}. Without both, no attacker can gain illegitimate access to an account. The Yubiduino lets users add this second layer of security to any service that supports Two Factor Authentication via the Time-based One Time Password standard, including Gmail, GitHub, and Facebook.


\section{System Design}
Hardware assembly was necessarily the first step in building the Yubiduino. The part list was fairly short:

\begin{itemize}
  \item Arduino Due microcontroller
  \item Breadboard
  \item Arduino-compatible Ethernet shield
  \item MicroSD card
  \item DS3231 real-time clock module with backup battery
  \item Momentary push-button switch
  \item \SI{10}{\kohm} resistor
  \item Various wires
\end{itemize}

% Figure 1
\begin{figure}[ht]
\centering
\includegraphics[width=\columnwidth]{button.png}
\caption{Button Circuit}
\label{fig:button}
\end{figure}

Assembly consisted of connecting the button as shown in Figure \ref{fig:button} and the real-time clock (RTC) module as shown in Figure \ref{fig:rtc}, then stacking the Ethernet shield directly on top of the Arduino.

% Figure 2
\begin{figure}[ht]
\centering
\includegraphics[width=\columnwidth]{rtc.jpg}
\caption{RTC Circuit}
\label{fig:rtc}
\end{figure}

As part of two-factor authorization setup, Gmail and Facebook will make available a base32-encoded secret key. To set up the Yubiduino, this key should be copied to a file named ``key.txt'' on the Yubiduino's FAT-formatted microSD card. This needs to be done only once. When the user logs in online, plugs in the Yubiduino, and activates it by pressing the button, it emulates a USB keyboard and ``types'' the digits of the one-time password into the currently-focused form field. The code itself was determined by running the TOTP algorithm using the shared secret key from the microSD card and the timestamp from the RTC.

\section{Implementation and Evaluation}

\subsection{Implementation}

% * Handling button press
% * Programming the RTC
% * Writing via the keyboard
% * Accessing SD card storage and reading key from file.
% * Base 32 decoding
% * Getting the time from the RTC
% * Fixing Sha library
% * Generating C to pass to HMAC-SHA1
% * Calling HMAC-SHA1
% * Truncating the result according to TOTP spec

Yubiduino's software was implemented in C++ using the AVR-GCC compiler to
target the Arduino Due board.  This board uses a 32-bit ARM core microcontroller running at \SI{84}{\MHz}.  This microcontroller allows for 4-byte wide data
operations to occur in one clock cycle, which provides significant performance
benefits for the cryptographic hashing code.

Several libraries were used in order to build the Yubiduino software package.
Standard libraries used were \texttt{Serial}, \texttt{Keyboard}, \texttt{SD},
and \texttt{SPI}.  In addition to these standard libraries, two third-party
libraries were used: \texttt{DS3231} and \texttt{Sha}.

The \texttt{Serial} library provides a way for the Arduino to communicate via a
serial interface.  This is primarily used for interacting with the Arduino
using the Arduino IDE's built in Serial Monitor. The \texttt{Keyboard} library
is used to allow the Arduino to send keyboard input to a connected computer.
The \texttt{SD} library is used to read and write files on an SD card and
supports both the FAT16 and FAT32 file systems.  An Ethernet shield was used in
order to provide an SD card input. The \texttt{SPI} library allows for access
to the Serial Peripheral Interface.  This interface allows for short distance
communication between microcontrollers.  This library was used to communicate
with the Real Time Clock (RTC) module over I\textsuperscript{2}C.

Third party libraries were used when standard library functionality was insufficient.  The \texttt{DS3231} library was used to communicate with the RTC.
It provided convenience methods for reading the current time and converting the
time between different formats.  The \texttt{Sha} library provided SHA1 and
HMAC-SHA1 cryptographic hashing capabilities.  In addition to this, a
\texttt{base32\_decode} function was used from the open source Google
Authenticator project.  This implementation of \texttt{base32\_decode} was
chosen to ensure that the secret key would be decoded in the exact way
intended.

The Arduino API specifies two methods that can be used as program entry points.
The \texttt{setup} method is run once any time the Arduino is powered on or
reset and is useful for performing setup and initialization tasks.  Yubiduino uses this method to
perform general setup including enabling input and output pins, initializing
third-party libraries, and several other actions outlined below.  The other
method is \texttt{loop}.  This method is run continuously as long as the
Arduino is powered on.  It is generally used to check for specific conditions
and run event handling routines when they are met.  For Yubiduino, this means
listening for a button press and generating a token.  If a button press is not
detected, the Yubiduino will sleep for a short amount of time in order to
conserve power and clock cycles.

The first step in implementing TOTP on the Arduino was to set the time on the
RTC.  This is necessary to make sure that the interval offset calculated
matches what the server expects.  Setting the time on the RTC is an operation
that only needed to be performed once.  In order to make this operation as user
friendly as possible, a serial utility was created.  The utility is invoked by
entering \texttt{set time} at the serial monitor while the Arduino is running.
The Arduino will recognize this as a valid command and prompt for the current
UTC year, month, day, hour, minute, and second.  Once this information is
entered the Arduino will use the \texttt{DS3231} library to communicate the
correct time to the RTC.  The serial utility can also be used to read the
current time by entering \texttt{time} at the serial monitor.

Once the time is set, the next piece of information needed is the secret key.
This is a 32-character string provided by the two-factor authorization service. This string is base32-encoded in order to prevent
confusion resulting from similar-looking letters and digits.  Yubiduino will
look for this key in a file on the SD card called key.txt.  The key can be
entered with or without spaces.  On startup, the key is read from the file and
decoded into a 20 byte string.  If the key file cannot be found, Yubiduino will
log an error to the serial monitor and refuse to start.  The serial utility can
also be used to view the key stored on the Yubiduino by entering \texttt{key}
at the serial monitor prompt.

After the \texttt{setup} method has completed, the \texttt{loop} method is
invoked.  This method continuously reads the \texttt{BUTTON\_PIN}.  When the
signal from the pin changes, it means the button has been pressed.  At this
point it allocates memory for a new token, which is stored in a 6-byte \texttt{char}
array.  It also gets the current time from the RTC and converts it into a UNIX
timestamp.  Yubiduino then passes the secret key, key length, current time, and
token buffer into the \texttt{totp} method.

The \texttt{totp} method performs the necessary calculations to turn the
current time and secret key into a token.  The first step in this process is to
calculate the time interval, \texttt{C}.  This is done by dividing the current
time by the interval value, which defaults to 30 seconds. \texttt{C} is
returned as a \texttt{long}, which is a 32-bit signed integer.  It must be
converted into a 64-bit unsigned integer before it can be hashed.  This is done
by creating an 8-byte array, setting the 4 most significant bytes to
\texttt{0x00}, and setting the remaining 4 bytes to the value stored in the
\texttt{long}.  The next step is to hash this value using HMAC-SHA1.  The
\texttt{Sha} library provides a convenient way of doing this.  The secret key
used for the hash is the same secret key set previously when setting up the
Yubiduino.  The resulting hash is returned as a 20-byte array, \texttt{H}.  The
next step in the process is to truncate \texttt{H} in such a way that the
resulting value is cryptographically sound. This is done by taking the 4 least
significant bits of \texttt{H}, called \texttt{O}.  4 bytes are taken from
\texttt{H}, starting at \texttt{O}.  The most significant bit is dropped, and
the remaining bits are stored as an unsigned 32-bit integer, \texttt{I}.  The
token is the lowest 6 digits of \texttt{I} represented in base 10.  If
\texttt{I} is fewer than 6 digits, it is left padded with 0s.

Once the token is calculated, it is sent to the user through the
\texttt{Keyboard} interface.  This allows the Yubiduino to fill in the correct
token without the user needing to type it themselves, preventing errors from
typos and timing.


\subsection{Evaluation}
For the Yubiduino's final test, we set up two-factor authentication on a Gmail account, saved the shared secret key to the microSD card, and successfully logged into the account on multiple occasions using the Yubiduino to generate and input the authentication code automatically. By this very real-world measure of using it on a popular production system, we consider the Yubiduino a success. Hashing speed was a particular concern going into the project; fortunately, it generates authentication codes apparently instantaneously. However, the Yubiduino is not without room for improvement: at its current size, it is impractical for everyday use, though we believe we have proven the concept and that miniaturization of the hardware is an achievable solution.

\section{Conclusion}
Users trust Internet companies to protect more private information than ever before. Two-factor authentication is a field-tested method that provides some much-needed additional account security. With the Yubiduino, we successfully designed and built an assistive device that implements the open, standardized Time-based One-Time Password algorithm, and we proved that success by using it to log into a real-world commercial system over the Internet.

% conference papers do not normally have an appendix

% use section* for acknowledgement

% trigger a \newpage just before the given reference
% number - used to balance the columns on the last page
% adjust value as needed - may need to be readjusted if
% the document is modified later
%\IEEEtriggeratref{8}
% The "triggered" command can be changed if desired:
%\IEEEtriggercmd{\enlargethispage{-5in}}

% references section

% can use a bibliography generated by BibTeX as a .bbl file
% BibTeX documentation can be easily obtained at:
% http://www.ctan.org/tex-archive/biblio/bibtex/contrib/doc/
% The IEEEtran BibTeX style support page is at:
% http://www.michaelshell.org/tex/ieeetran/bibtex/
\nocite{totp,
        mfa,
        totp_david,
        arduino_due,
        serial,
        mouse,
        strong_mfa,
        keyboard,
        sd,
        spi,
        rfc,
        authy,
        github}
\bibliographystyle{IEEEtran}
% argument is your BibTeX string definitions and bibliography database(s)
\bibliography{IEEEabrv,./report}
%
% <OR> manually copy in the resultant .bbl file
% set second argument of \begin to the number of references
% (used to reserve space for the reference number labels box)
%\begin{thebibliography}{1}

%\bibitem{IEEEhowto:kopka}
%H.~Kopka and P.~W. Daly, \emph{A Guide to \LaTeX}, 3rd~ed.\hskip 1em plus
%  0.5em minus 0.4em\relax Harlow, England: Addison-Wesley, 1999.

%\end{thebibliography}


% that's all folks
\end{document}
